%%%%%%%%%%%%%%%%%%%%%%%%%%%%%%%%%%%%%%%%%%%%%%%%%%%%%%%%%%%%%%%%%%%%
%% LaTeX Template Version 2.3
%%
%% Nathan Lowrie
%% Sandia National Laboratories
%%
%% (C)2007 - Free for personal use, please email me improvements to
%% solodex2151@gmail.com
%%%%%%%%%%%%%%%%%%%%%%%%%%%%%%%%%%%%%%%%%%%%%%%%%%%%%%%%%%%%%%%%%%%%

% Preamble
    % Packages included
    \documentclass[letterpaper,10pt]{RevisedBook}
    \usepackage{ifthen}
    \usepackage[ansinew]{inputenc}
    \usepackage{graphics}
    \usepackage{fancyhdr}
    \usepackage{amsmath}	
    \usepackage{graphicx}
    
    \oddsidemargin .5in
    \evensidemargin .5in
    \textwidth 5.5in
    
    \pagestyle{fancy}
    \sloppy
    \nonfrenchspacing
    \renewcommand{\baselinestretch}{1.0}
    \clubpenalty=9999    % Not higher!
    \widowpenalty=9999   % Not higher!
    \setcounter{secnumdepth}{4}
    \setcounter{tocdepth}{3}
%    \makeindex
    \addtolength{\skip\footins}{5mm}

% Some custom macros
    % Titlerule is a FAT ruler
    \newcommand{\titlerule}{\rule{\linewidth}{1.5mm}}

    % For comments in the draft - work in progress
    \newcommand{\betainsert}[2]{\fbox{#1}\marginnote{\textsf{#2}}}

    % Notes in the margin are nicer this way.
    \newcommand{\marginnote}[1]{\marginpar{\scriptsize\raggedright #1}}

% Do you want to have the possibility of including color in your PDFs?
    \usepackage{color}

% Check if in PDFLaTeX or ``normal''
    \newif\ifpdf
      \ifx\pdfoutput\undefined
      \pdffalse
    \else
      \pdfoutput=1
      \pdftrue
    \fi

%Define variables
\renewcommand{\title}{The Dabo Book}				%Title Goes Here
\newcommand{\subtitle}{3-Tier Applications Made Easy}	%Subtitle Goes Here
\renewcommand{\author}{Paul McNett, Nathan Lowrie}					%Author Goes Here
\newcommand{\copyrightHolder}{Ed Leafe, Paul McNett, et. al.}	%Copyright Holder Goes Here
\newcommand{\subject}{3-tier desktop applications}					%Subject Goes Here
\newcommand{\keywords}{Dabo, Python, database}					%Keywords go here
    
% Set up PDF specific. These are used when the file is compiled with 
% pdflatex instead of ordinary latex.
    \ifpdf
      % if PDFLaTeX use these parameters
      \usepackage[pdftex,colorlinks=true,urlcolor=blue,pdfstartview=FitH]{hyperref}
      \pdfcompresslevel=9
      \hypersetup{
			pdftitle={\title},			%Enter title here
			pdfauthor={\author},				%Enter Author here
			pdfsubject={\subject},		%Enter Subject here
			pdfkeywords={\keywords}		%Enter Keywords here
      } 
    \else
      % Ordinary tex use these parameters
      \usepackage{hyperref}
    \fi
   
\begin{document}

\frontmatter									% only in book class (roman page #s)

% Making a nice TITLEPAGE
\begin{titlepage}
    \thispagestyle{empty}
    \begin{center}
%    \includegraphics[width= 8cm]{logo}			%Uncomment if you want a logo graphic
    \end{center}
    \vspace*{\stretch{1}}
    \begin{center}
        \titlerule\\[3mm]
        \Huge \textsc{\title\\\subtitle}\\[5mm]		%Enter Title and subtitle here
        \begin{center}
		\huge \author\\[3.5mm]						%Enter company or person authoring
    	\end{center}
       	\titlerule\\
    \end{center}
    \scriptsize \today
    \vspace*{\stretch{2}}
    \begin{center}
        \textcopyright 2004-2007 \copyrightHolder			%Enter company or person for copyright here
    \end{center}
\end{titlepage}

\pagestyle{fancy}                         %Forces the page to use the fancy template

% Setting up pagestyles for ``fancy''
\setlength\headheight{23.11pt}
\renewcommand{\chaptermark}[1]{\markboth{\emph{#1}}{}}%\textbf{Chapter \thechapter}:\ \emph{#1}}{}}
\renewcommand{\sectionmark}[1]{\markright{\thesection\ \boldmath\textbf{#1}\unboldmath}}
                                          %The text used in the header is determined by the arguments to the \markboth
                                          % and \markright commands used here. The chapter information will appear as the
                                          % chapter number in bold, followed by a dot and a space, followed by the chapter
                                          % title (dealt with by LaTeX---no user intervention needed here) in italics. The
                                          % section information will appear as the section number, followed by a space, and
                                          % then the section title (again generated automatically) in bold. Any maths in
                                          % the section title will also appear in bold (provided the bold font exists).
\fancyhf{}                                %Clears all header and footer fields, in preparation.
\fancyhead[LO, RE]{\textbf{\title}\\\textbf{\subtitle}}
                                          %Displays the Title and subtitle the header,
                                          % to the left on odd pages and to the right on even pages.
\fancyhead[LE, RO]{\nouppercase{\leftmark}}   %Displays the upper-level (chapter) information---
                                          % as determined above---in non-upper case in the header, to the left on even pages.
						                  %Displays the lower-level (section) information---as
                                          % determined above---in the header, to the right on odd pages.
% \fancyfoot[c]{\textcopyright\ 2004-2007 \copyrightHolder}
\fancyfoot[LE,RO]{\thepage}
\fancyfoot[LO,RE]{\today}
\renewcommand{\headrulewidth}{1.3pt}      %Underlines the header. (Set to 0pt if not required).
\renewcommand{\footrulewidth}{1.3pt}      %Underlines the footer. (Set to 0pt if not required).

\tableofcontents
\newpage

% --- DOCUMENT START ---


% Preface for the dabo book

\chapter{Preface}

Microsoft purchased Fox Software and gained the knowledge base to produce the 
best user-friendly application development suites in the industry. Using Visual Basic, 
Access, .NET, COM, MSDE and FoxPro, hundreds of thousands of developers and 
consultants world-wide were able to provide solutions for their clients, from 
shrink-wrapped applications to completely custom programs.

For over a decade, the situation was mutually beneficial: the large company got to 
make billions off its operating system, and the developer got to make a living selling 
solutions using low-priced development tools provided by that same large company.

Something started happening, however, late in the 1990's. At a time when everyone 
thought Unix was uttering its final words, "Linux" became a household term. A 
student's hobby in 1991 became a world-class enterprise operating system by 1998, 
and since then Linux has made impressive gains in the user application space: the 
desktop. At about the same time, Apple Computer released Darwin - another Unix 
based operating system - to the open source world and built a rock-solid user 
friendly interface on top of it, and today Apple's successful future seems guaranteed: 
elegant, user-friendly applications built on top of Darwin/OS X, such as iTunes, 
iPhoto, iDVD, and iMovie are reverberating among a growing user base.

A sea change is underway. Up until now, and into 2006, the open source community 
has been playing catch-up with Microsoft, trying to make applications that do 
everything the typical Microsoft Windows user expects, in the same ubiquitous 
fashion. As I type this (September 2004) I can say that the open source community 
has almost caught up. OpenOffice.org is approaching version 2.0, and can already 
replace Microsoft Office for all but the most complex Word, Powerpoint, and Excel 
documents. Mozilla Firefox and Thunderbird are approaching 1.0, and already 
handle browsing and emailing better than Microsoft Internet Explorer and Outlook. 
The Gnome and KDE Linux desktops, besides being fast and stable, now have all 
the goodies any modern user would want. From now on, the catch-up phase is over, 
and real innovation will begin to be defined not by one huge company, but by a 
loosely-knit community of open source developers from all over the world. The big 
company will have to become a part of that community - and play by its rules - or 
risk being sucked into the undertow.

We aren't quite there yet. One very important missing piece for software development 
on Linux or Mac is a comprehensive, easy-to-use, flexible Integrated Development 
Environment (IDE), such as Microsoft Visual Basic, that lets people develop powerful 
applications without necessarily having to know how to write great code. An IDE is 
basically a collection of power tools, such as a visual form designer, a graphical 
debugger, a project manager, and a help system. Until such an application becomes 
available, the only people developing open source applications will be people 
comfortable with the command line, make files, and a c compiler.

Enter Dabo, an open source, data aware, 3-tier development framework that you 
can use to develop open source and/or proprietary applications for distribution to 
your customers. Dabo aims to be easy to learn, fun to work with, flexible, and 
powerful. You can program in Python by hand using any editor, or you can use the 
Dabo IDE which centralizes all the files in your project and offers all the power tools 
you need to create your databases, build your user interface, write your business 
rules, and create your reports (printouts or previews of your data, formatted the 
way you define).

%Not really sure what this is so I commented it out.
%Intro paragraph - first paragraph style

\section{Target Audience}

The Dabo Book assumes some knowledge of a typical Microsoft-like IDE such as 
Visual Basic or Visual FoxPro, but that isn't required to get by. All that is really 
needed is the desire to create a killer application. Follow along with the examples, 
and you'll know all you need to know in no time at all.

If you happen to have experience with the Python programming language, which 
Dabo is built on top of, that is great but again not required.

In a nutshell, this book is for anyone wishing to develop a desktop application that 
will run on Linux, Macintosh, and Windows, and that will connect to an external 
database to get and store data.

\section{Organization of this Book}

The book is organized into several parts:

\subsection{Context and Introduction}

Covers the history of Dabo as well as its features, architecture, components, and 
install methods. Also includes an introduction to the Python programming language.

\subsection{Dabo Overview and Patterns}

Explains the components that make up Dabo, shows some simple examples, 
introduces the new user to the Python language, and goes into some detail on the 
"Three Tiers of Dabo". Also shows the typical structure of a Dabo application, and 
the typical workflow patterns of creating a Dabo application.

\subsection{Tutorials}

Here is where the fun begins. Follow along with the design and implementation of a 
"real-world" data-aware application, from start to finish. From the structure of the
database, to the writing of the business rules, to the designing of the screens and 
application framework, to debugging and documenting, to source control, this is the 
real meat of the book. Get through this and you'll know how to start making your 
own application.

\subsection{API Reference}

You won't necessarily want to read this part of the book from beginning to end, 
but it'll probably be well thumbed through, as it'll be your reference to all the 
properties, methods, and events available for your use while developing your 
application.

\subsection{Appendices}

A number of stand-alone articles are presented here, including an in-depth tour of 
the Dabo Report Designer, debugging tactics, visual design options, how to contribute 
to the development of Dabo, and more.


\fancyhead[RO]{Chapter \thechapter}

\mainmatter                             % only in book class (arabic page #s)
\newpage

%Put your parts and chapters here
% Will house all of the introduction components
% Each chapter gets it's own file

\part{Introduction}

%What is dabo chapter

\chapter{What is Dabo?}

Dabo provides an abstraction layer for a variety of open source projects, for the 
purpose of providing a solid and flexible framework for developing multiplatform 
data-aware business applications. User/developers can use the powerful Python 
programming language to write their business logic and lay out their user-interface 
elements, harnessing the Dabo framework and thus not getting preoccupied with the 
implementation details.

\section{3-Tier}

Dabo provides a 3-tier approach to application design, separating database access 
from business rules from user-interface layout. Dabo also provides an Application 
object that provides common functions and controls the event loop.

Dabo allows you to use each tier independently, for instance only using the database 
tier for a simple script, or only using the UI tier for a simple GUI app that doesn't 
need database access. But those use-cases will be limited. In a typical Dabo 
application, 90\% of the user code will end up in the business tier, using subclasses 
of the Dabo Business Object, 0\% in the database tier, and the rest as layout code 
in the user-interface tier.

Dabo's tiers are related in a chain-of-responsibility pattern, so that when a user 
chooses, for example, to save their changes, the user-interface will communicate 
that request to the business object, which will validate the request against the 
business rules, and if all rules pass the request will go to the database tier to 
actually save the change to the database. If the change fails, such as in the 
business object for validation reasons, or in the database tier for connectivity 
reasons, the exception gets propagated back up to the user-interface tier and the 
user is notified.

\section{Flexible Database Support}

Dabo supports all databases for which there is a Python wrapper that conforms to 
the dbapi version 2. This includes all popular databases, as illustrated in (need 
fig-tier1-1). Please note that this book is being written for future benefit, and that 
as of this writing the only supported databases are MySQL, Firebird, and Sqlite, and 
MySQL is the most tested.

\section{Flexible User Interface Support}

After version 1.0, Dabo will support a selection of user-interface libraries, as 
illustrated in (need fig-tier3-1). The support of multiple libraries while allowing the 
Dabo developer to use a common API makes Dabo a very flexible, powerful solution. 
Different toolkits have their pros and cons; you can choose which one to deploy 
and not worry too much about that during development. You could conceivably do 
all your development using one toolkit and deploy with another. You may have one 
deployment using PyQt, and another using wxPython, both using the same codebase. 
Please note that this book is being written for future benefit, and that as of this 
writing the only supported user interface is wxPython.

\section{Cross-platform Support}

Dabo is truly multi-platform. Develop on any supported platform, and deploy the 
same code base to any supported platform. The supported platforms are Macintosh 
OS X (10.2 or higher), Linux, and Windows (98SE or higher). (need 
fig-multiplatform-lin), (need fig-multiplatform-osx), and (need fig-multiplatform-win) 
show the same Dabo-developed application running on all three platforms.

%History of Dabo Chapter

\chapter{History of Dabo}

Dabo is the result of a few years of research, starting in 2001 when I started taking an active interest in the Linux operating system and open source software in general. I had been using Microsoft Visual FoxPro to develop data-aware business applications for my clients, and with mixed messages coming from Microsoft and the FoxPro community as to the long-term viability of FoxPro as a product, I started looking for alternatives from the open source community, alternatives that would permit the development of powerful database applications for multi-platform deployment.

My quest led me first to Borland Delphi, which had just recently announced a prerelease version of Kylix, the Linux version of Delphi. This product would allow deployment of a semi-common codebase to one flavor of Linux (RedHat 7.x) and Windows. It had good data-aware controls, but only in the Enterprise version. The execution performance was pitiful, and code wasn't truly portable between platforms. The lack of a commitment from Borland to support Macintosh was the straw that made me look elsewhere.

The next stop was to take a serious look at Java, which does have a clean and elegant language and does run pretty much equally on all platforms, thanks to the Java Virtual Machine. I developed some prototype applications using the Swing components which performed equally horribly on all three platforms. But, it was easy development and deployment, although database integration was pretty obtuse compared to what I was used to in Visual FoxPro. My overall feeling was that Java may really take over the world, but it has a long way to go performance-wise, and it really isn't fun to code.

Somewhere about this time, I found out that Kylix was using a toolkit called Qt to provide the user interface components, and that Kylix was using the last-generation (Qt 2.x) instead of the newer, much nicer Qt 3.x version. I downloaded the GPL'd version of Qt for Linux, and followed some tutorials, and was able to build some very impressive C++ applications that performed very well. I never got my head around C++, however, so I felt like I'd be painting myself into a corner by pursuing this angle. Also, database support required the Enterprise version of Qt, which is something like \$1200 per platform. Hardly free and open source.

Eventually, I came to know of a programming language called Python, which Ed Leafe had been using to power his website using a product called Zope. I found Python to be very intuitive to learn, with an easy readable syntax not unlike FoxPro. Python, on top of being free and open source, also comes with "batteries included", meaning that most everything you'd want to do comes in the standard library, including building user interfaces. Python also comes with three native data types that have to do with sequences - in other words, Python can represent database tables and fields natively. Come to find out, Python also provides an API for connecting to all kinds of database servers. In other words, Python comes with a lot of the pieces I'd been searching for over the past few years.

While it was great finding that Python had a lot of the pieces to my puzzle, it was another thing entirely realizing that putting all the pieces together into a workable whole would prove to be anything but easy. Yes, Python can connect to any database and retrieve and update data. Yes, Python has a graphical user interface. No, connecting to a given database isn't the same as connecting to a different database. And no, the user interface that comes with Python is not very modern.

In the Fall of 2003, I set out to create a framework for developing data-aware applications in Python, using a GUI toolkit I'd just learned about called wxWindows. I'd actually heard about wxWindows before, but had discounted it thinking that it was only for the Windows platform. They've since changed their name to wxWidgets at Microsoft's request, which may keep future developers from being as confused as I was. I named this framework 'Dabo', because it sounded fun and reminds me of words like 'data', 'business', 'application', and 'objects'. Also, we were watching Star Trek: Deep Space Nine at the time and I liked the Dabo Girls.

I ended up learning the wxWidgets toolkit pretty well, but created a pile of spaghetti code that had no separation between the database, the business rules, and the user interface. It was really a mess and completely unmaintainable. I set the project down for a few months, until I was contacted in March of 2004 by Ed Leafe, a long-time FoxPro guru that was looking for ways to move his skillset over to open source, multiplatform development - he was looking for the same things I'd been seeking over the previous couple years. I decided to share my code with him, along with a sample application, and he very diplomatically explained all the design problems with my approach. We came up with an agreement to redesign Dabo from the ground up, with a 3-tier model.

By May of 2004 we announced our work to the public, got a website and mailing lists, and encouragement from diverse areas of the open source, Python, and FoxPro communities. As I write this in September 2004, Dabo is under active development, the user interface is 80\% abstracted, 3 databases are supported, and a user-interface graphical designer is underway. It is already possible to create powerful data-aware applications and to deploy them to Windows, Linux, and Macintosh.

%Chapter for installing and setting up dabo

\chapter{Installation of the Dabo Development Environment}

This chapter details the process for getting your computer set up for developing applications using Dabo. While these instructions would work for your deployment targets as well, there are better ways to deploy applications than by using the instructions here. Dabo has dependencies on a number of external libraries, and while developing your applications you'll want to keep all those libraries - and Dabo - as current as possible. For deployment, you want better control over the versions in use. Deploying applications is covered in (need link).

\section{Downloading and Installing}

Dabo has dependencies on a number of external libraries, and those dependencies will vary depending on your choice of database and user-interface library. In general, you will want to install, in this order:
\begin{enumerate}
	\item Python (the most recent stable version available). Python is available for download from http://www.python.org. Follow the instructions there for installation instructions1 for your platform. Python is certainly already installed on your Linux or Macintosh system, but for Windows you may find that you need to install it yourself. No matter what, please check to see if the version you have is relatively recent and if not you should download and install the most recent stable Python release.

	\item MySQL Client.2 MySQL is available for download from http://www.mysql.com. Download the most recent stable version available. If you already have a MySQL server installed somewhere on your network available to you (and you have the rights to create and drop databases), only the client is required. Otherwise, also install the server so that you'll be able to follow along with some later examples.

	\item MySQLdb. This is the Python db-api wrapper that allows your Python code to talk to the MySQL client library. This is an easy install using the standard Python distutils3 method. Download MySQLdb from http://sourceforge.net/projects/mysql-python. Get the latest stable version.

	\item wxPython. This is the standard user-interface toolkit for Dabo, and at the time of this writing is required for building applications that present an interface to the user. In the future, Dabo will support other user-interface toolkits as well, but for now the only supported toolkit is wxPython. wxPython is in a state of rapid development, so it is best to stay as current as you can with it. Download and install the most recent stable version from http://www.wxpython.org.

	\item Dabo. Get the most recent version of Dabo from http://dabodev.com. Be sure to get the main Dabo package as well as dabodemo and daboIDE. Like MySQLdb, Dabo uses distutils so a simple python setup.py install should get Dabo into your Python installation's site-packages directory, which is where all third-party libraries for Python are normally installed.
\end{enumerate}

\section{Testing Your Installation}

Now that you've downloaded and installed all the prerequisites, you need to run some tests to be reasonably sure everything is installed correctly. The tests involve interacting with your operating system's command line, which as a developer you really should try to get familiar with.

Microsoft Windows: Go to Start|Run and type 'cmd' <enter>.
Apple Macintosh: Navigate to your Applictions/Utilities directory and double-click on the Terminal application.
Linux/UNIX: Different distributions put this in different places. Look for xterm, terminal, or command-line in your desktop menu system.
	
Open up your command line, and type python. You should get output like:	%NEED TO GET SETUP FOR STANDARD CODE OUTPUT SETUP
\begin{verbatim}
[pmcnett@sol book]$ python
Python 2.3.2 (#1, Oct  6 2003, 10:07:16)
[GCC 3.2.2 20030222 (Red Hat Linux 3.2.2-5)] on linux2
Type "help", "copyright", "credits" or "license" for more information.
>>>
You are now inside Python's command interpreter. Test to make sure that MySQLdb, wxPython, and Dabo load correctly. If there are no errors, they are installed correctly.

>>> import MySQLdb
>>> import wx
>>> import dabo
Dabo Info Log: Thu Sep  9 19:16:23 2004: No default UI set. (DABO_DEFAULT_UI)
>>>
\end{verbatim}
The message from Dabo is normal, and no errors happened during the import of the other packages, so everything is set up correctly on my system. 

% \section{Summary}

% Dabo is a framework built on Python that provides a clean API for developers to build data-aware business applications that are cross-platform. In addition to this underlying framework, Dabo also provides some power tools, such as a visual UI designer based on wxGlade, for designing and laying out your forms, menus, and other UI elements, and wizards and demo applications for getting started. These power tools are discussed elsewhere in this book.


%This section reserved for bibliography and index
\backmatter

%\include{bibliography}  % include bibliography
%\include{index}         % include index

\end{document}                          % The required last line
