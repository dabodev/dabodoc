% Preface for the dabo book

\chapter{Preface}

Microsoft purchased Fox Software and gained the knowledge base to produce the best user-friendly application development suites in the industry. Using Visual Basic, Access, .NET, COM, MSDE and FoxPro, hundreds of thousands of developers and consultants world-wide were able to provide solutions for their clients, from shrink-wrapped applications to completely custom programs.

For over a decade, the situation was mutually beneficial: the large company got to make billions off its operating system, and the developer got to make a living selling solutions using low-priced development tools provided by that same large company.

Something started happening, however, late in the 1990's. At a time when everyone thought Unix was uttering its final words, "Linux" became a household term. A student's hobby in 1991 became a world-class enterprise operating system by 1998, and since then Linux has made impressive gains in the user application space: the desktop. At about the same time, Apple Computer released Darwin - another Unix based operating system - to the open source world and built a rock-solid user friendly interface on top of it, and today Apple's successful future seems guaranteed: elegant, user-friendly applications built on top of Darwin/OS X, such as <application>iTunes</application>, <application>iPhoto</application>, <application>iDVD</application>, and <application>iMovie</application> are reverberating among a growing user base.

A sea change is underway. Up until now, and into 2006, the open source community has been playing catch-up with Microsoft, trying to make applications that do everything the typical Microsoft Windows user expects, in the same ubiquitous fashion. As I type this (September 2004) I can say that the open source community has almost caught up. <application>OpenOffice.org</application> is approaching version 2.0, and can already replace <application>Microsoft Office</application> for all but the most complex <application>Word</application>, <application>Powerpoint</application>, and <application>Excel</application> documents. <application>Mozilla FireFox</application> and <application>ThunderBird</application> are approaching 1.0, and already handle browsing and emailing better than <application>Microsoft Internet Explorer</application> and <application>Outlook</application>. The Gnome and KDE Linux desktops, besides being fast and stable, now have all the goodies any modern user would want. From now on, the catch-up phase is over, and real innovation will begin to be defined not by one huge company, but by a loosely-knit community of open source developers from all over the world. The big company will have to become a part of that community - and play by its rules - or risk being sucked into the undertow.

We aren't quite there yet. One very important missing piece for software development on Linux or Mac is a comprehensive, easy-to-use, flexible Integrated Development Environment (IDE), such as Microsoft Visual Basic, that lets people develop powerful applications without necessarily having to know how to write great code. An IDE is basically a collection of power tools, such as a visual form designer, a graphical debugger, a project manager, and a help system. Until such an application becomes available, the only people developing open source applications will be people comfortable with the command line, make files, and a c compiler.

Enter Dabo, an open source, data aware, 3-tier development framework that	you can use to develop open source and/or proprietary applications for distribution to your customers. Dabo aims to be easy to learn, fun to work with, flexible, and powerful. You can program in Python by hand using any editor, or you can use the Dabo IDE which centralizes all the files in your project and offers all the power tools you need to create your databases, build your user interface, write your business rules, and create your reports (printouts or previews of your data, formatted the way you define).

%Not really sure what this is so I commented it out.
%Intro paragraph - first paragraph style

\section{Target Audience}

The Dabo Book assumes some knowledge of a typical Microsoft-like IDE such as Visual Basic or Visual FoxPro, but that isn't required to get by. All that is really needed is the desire to create a killer application. Follow along with the examples, and you'll know all you need to know in no time at all.

If you happen to have experience with the Python programming language, which Dabo is built on top of, that is great but again not required.

In a nutshell, this book is for anyone wishing to develop a desktop application that will run on Linux, Macintosh, and Windows, and that will connect to an external database to get and store data.

\section{Organization of this Book}

The book is organized into several parts:

\subsection{Context and Introduction}

Covers the history of Dabo as well as its features, architecture, components, and install methods. Also includes an introduction to the Python programming language.

\subsection{Dabo Overview and Patterns}

Explains the components that make up Dabo, shows some simple examples, introduces the new user to the Python language, and goes into some detail on the "Three Tiers of Dabo". Also shows the typical structure of a Dabo application, and the typical workflow patterns of creating a Dabo application.

\subsection{Tutorials}

Here is where the fun begins. Follow along with the design and implementation of a "real-world" data-aware application, from start to finish. From the structure of the database, to the writing of the business rules, to the designing of the screens and application framework, to debugging and documenting, to source control, this is the real meat of the book. Get through this and you'll know how to start making your own application.

\subsection{API Reference}

You won't necessarily want to read this part of the book from beginning to end, but it'll probably be well thumbed through, as it'll be your reference to all the properties, methods, and events available for your use while developing your application.

\subsection{Appendices}

A number of stand-alone articles are presented here, including an in-depth tour of the Dabo Report Designer, debugging tactics, visual design options, how to contribute to the development of Dabo, and more.
