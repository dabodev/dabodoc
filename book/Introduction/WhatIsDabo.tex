%What is dabo chapter

\chapter{What is Dabo?}

Dabo provides an abstraction layer for a variety of open source projects, for the 
purpose of providing a solid and flexible framework for developing multiplatform 
data-aware business applications. User/developers can use the powerful Python 
programming language to write their business logic and lay out their user-interface 
elements, harnessing the Dabo framework and thus not getting preoccupied with the 
implementation details.

\section{3-Tier}

Dabo provides a 3-tier approach to application design, separating database access 
from business rules from user-interface layout. Dabo also provides an Application 
object that provides common functions and controls the event loop.

Dabo allows you to use each tier independently, for instance only using the database 
tier for a simple script, or only using the UI tier for a simple GUI app that doesn't 
need database access. But those use-cases will be limited. In a typical Dabo 
application, 90\% of the user code will end up in the business tier, using subclasses 
of dBizobj, 0\% in the database tier, and the rest as layout code 
in the user-interface tier.

Dabo's tiers are related in a chain-of-responsibility pattern, so that when a user 
chooses, for example, to save their changes, the user-interface will communicate 
that request to the business object, which will validate the request against the 
business rules, and if all rules pass the request will go to the database tier to 
actually save the change to the database. If the change fails, such as in the 
business object for validation reasons, or in the database tier for connectivity 
reasons, the exception gets propagated back up to the user-interface tier and the 
user is notified.

\section{Flexible Database Support}

Dabo supports all databases for which there is a Python wrapper that conforms to 
the dbapi version 2. This includes MySQL, PostgreSQL, SQLite, Firebird, and 
Microsoft SQL Server. 

\section{Flexible User Interface Support}

After version 1.0, Dabo will support a selection of user-interface libraries, as 
illustrated in (need fig-tier3-1). The support of multiple libraries while allowing the 
Dabo developer to use a common API makes Dabo a very flexible, powerful solution. 
Different toolkits have their pros and cons; you can choose which one to deploy 
and not worry too much about that during development. You could conceivably do 
all your development using one toolkit and deploy with another. You may have one 
deployment using PyQt, and another using wxPython, both using the same codebase. 
Please note that this book is being written for future benefit, and that as of this 
writing the only supported user interface is wxPython.

\section{Cross-platform Support}

Dabo is truly multi-platform. Develop on any supported platform, and deploy the 
same code base to any supported platform. The supported platforms are Macintosh 
OS X (10.2 or higher), Linux, and Windows (98SE or higher). (need 
fig-multiplatform-lin), (need fig-multiplatform-osx), and (need fig-multiplatform-win) 
show the same Dabo-developed application running on all three platforms.
